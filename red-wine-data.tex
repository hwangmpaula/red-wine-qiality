\documentclass[]{article}
\usepackage{lmodern}
\usepackage{amssymb,amsmath}
\usepackage{ifxetex,ifluatex}
\usepackage{fixltx2e} % provides \textsubscript
\ifnum 0\ifxetex 1\fi\ifluatex 1\fi=0 % if pdftex
  \usepackage[T1]{fontenc}
  \usepackage[utf8]{inputenc}
\else % if luatex or xelatex
  \ifxetex
    \usepackage{mathspec}
  \else
    \usepackage{fontspec}
  \fi
  \defaultfontfeatures{Ligatures=TeX,Scale=MatchLowercase}
\fi
% use upquote if available, for straight quotes in verbatim environments
\IfFileExists{upquote.sty}{\usepackage{upquote}}{}
% use microtype if available
\IfFileExists{microtype.sty}{%
\usepackage{microtype}
\UseMicrotypeSet[protrusion]{basicmath} % disable protrusion for tt fonts
}{}
\usepackage[margin=1in]{geometry}
\usepackage{hyperref}
\hypersetup{unicode=true,
            pdfborder={0 0 0},
            breaklinks=true}
\urlstyle{same}  % don't use monospace font for urls
\usepackage{longtable,booktabs}
\usepackage{graphicx,grffile}
\makeatletter
\def\maxwidth{\ifdim\Gin@nat@width>\linewidth\linewidth\else\Gin@nat@width\fi}
\def\maxheight{\ifdim\Gin@nat@height>\textheight\textheight\else\Gin@nat@height\fi}
\makeatother
% Scale images if necessary, so that they will not overflow the page
% margins by default, and it is still possible to overwrite the defaults
% using explicit options in \includegraphics[width, height, ...]{}
\setkeys{Gin}{width=\maxwidth,height=\maxheight,keepaspectratio}
\IfFileExists{parskip.sty}{%
\usepackage{parskip}
}{% else
\setlength{\parindent}{0pt}
\setlength{\parskip}{6pt plus 2pt minus 1pt}
}
\setlength{\emergencystretch}{3em}  % prevent overfull lines
\providecommand{\tightlist}{%
  \setlength{\itemsep}{0pt}\setlength{\parskip}{0pt}}
\setcounter{secnumdepth}{0}
% Redefines (sub)paragraphs to behave more like sections
\ifx\paragraph\undefined\else
\let\oldparagraph\paragraph
\renewcommand{\paragraph}[1]{\oldparagraph{#1}\mbox{}}
\fi
\ifx\subparagraph\undefined\else
\let\oldsubparagraph\subparagraph
\renewcommand{\subparagraph}[1]{\oldsubparagraph{#1}\mbox{}}
\fi

%%% Use protect on footnotes to avoid problems with footnotes in titles
\let\rmarkdownfootnote\footnote%
\def\footnote{\protect\rmarkdownfootnote}

%%% Change title format to be more compact
\usepackage{titling}

% Create subtitle command for use in maketitle
\newcommand{\subtitle}[1]{
  \posttitle{
    \begin{center}\large#1\end{center}
    }
}

\setlength{\droptitle}{-2em}

  \title{}
    \pretitle{\vspace{\droptitle}}
  \posttitle{}
    \author{}
    \preauthor{}\postauthor{}
    \date{}
    \predate{}\postdate{}
  

\begin{document}

\section{RED WINE DATA by Paula
Hwang}\label{red-wine-data-by-paula-hwang}

This report explores a dataset containing quality and attributes for
approximately 1600 wines.

\begin{verbatim}
## [1] "C:/Users/Alexander Smith/Desktop/wine-data"
\end{verbatim}

Our dataset consists of 13 variables, with almost 1599 observations.

\paragraph{Display variables and
observations}\label{display-variables-and-observations}

\begin{verbatim}
## [1] 1599   13
\end{verbatim}

\paragraph{Display the summary of the
data}\label{display-the-summary-of-the-data}

\begin{verbatim}
## 'data.frame':    1599 obs. of  13 variables:
##  $ X                   : int  1 2 3 4 5 6 7 8 9 10 ...
##  $ fixed.acidity       : num  7.4 7.8 7.8 11.2 7.4 7.4 7.9 7.3 7.8 7.5 ...
##  $ volatile.acidity    : num  0.7 0.88 0.76 0.28 0.7 0.66 0.6 0.65 0.58 0.5 ...
##  $ citric.acid         : num  0 0 0.04 0.56 0 0 0.06 0 0.02 0.36 ...
##  $ residual.sugar      : num  1.9 2.6 2.3 1.9 1.9 1.8 1.6 1.2 2 6.1 ...
##  $ chlorides           : num  0.076 0.098 0.092 0.075 0.076 0.075 0.069 0.065 0.073 0.071 ...
##  $ free.sulfur.dioxide : num  11 25 15 17 11 13 15 15 9 17 ...
##  $ total.sulfur.dioxide: num  34 67 54 60 34 40 59 21 18 102 ...
##  $ density             : num  0.998 0.997 0.997 0.998 0.998 ...
##  $ pH                  : num  3.51 3.2 3.26 3.16 3.51 3.51 3.3 3.39 3.36 3.35 ...
##  $ sulphates           : num  0.56 0.68 0.65 0.58 0.56 0.56 0.46 0.47 0.57 0.8 ...
##  $ alcohol             : num  9.4 9.8 9.8 9.8 9.4 9.4 9.4 10 9.5 10.5 ...
##  $ quality             : int  5 5 5 6 5 5 5 7 7 5 ...
\end{verbatim}

\paragraph{Display a summary another
way}\label{display-a-summary-another-way}

\begin{verbatim}
##        X          fixed.acidity   volatile.acidity  citric.acid   
##  Min.   :   1.0   Min.   : 4.60   Min.   :0.1200   Min.   :0.000  
##  1st Qu.: 400.5   1st Qu.: 7.10   1st Qu.:0.3900   1st Qu.:0.090  
##  Median : 800.0   Median : 7.90   Median :0.5200   Median :0.260  
##  Mean   : 800.0   Mean   : 8.32   Mean   :0.5278   Mean   :0.271  
##  3rd Qu.:1199.5   3rd Qu.: 9.20   3rd Qu.:0.6400   3rd Qu.:0.420  
##  Max.   :1599.0   Max.   :15.90   Max.   :1.5800   Max.   :1.000  
##  residual.sugar     chlorides       free.sulfur.dioxide
##  Min.   : 0.900   Min.   :0.01200   Min.   : 1.00      
##  1st Qu.: 1.900   1st Qu.:0.07000   1st Qu.: 7.00      
##  Median : 2.200   Median :0.07900   Median :14.00      
##  Mean   : 2.539   Mean   :0.08747   Mean   :15.87      
##  3rd Qu.: 2.600   3rd Qu.:0.09000   3rd Qu.:21.00      
##  Max.   :15.500   Max.   :0.61100   Max.   :72.00      
##  total.sulfur.dioxide    density             pH          sulphates     
##  Min.   :  6.00       Min.   :0.9901   Min.   :2.740   Min.   :0.3300  
##  1st Qu.: 22.00       1st Qu.:0.9956   1st Qu.:3.210   1st Qu.:0.5500  
##  Median : 38.00       Median :0.9968   Median :3.310   Median :0.6200  
##  Mean   : 46.47       Mean   :0.9967   Mean   :3.311   Mean   :0.6581  
##  3rd Qu.: 62.00       3rd Qu.:0.9978   3rd Qu.:3.400   3rd Qu.:0.7300  
##  Max.   :289.00       Max.   :1.0037   Max.   :4.010   Max.   :2.0000  
##     alcohol         quality     
##  Min.   : 8.40   Min.   :3.000  
##  1st Qu.: 9.50   1st Qu.:5.000  
##  Median :10.20   Median :6.000  
##  Mean   :10.42   Mean   :5.636  
##  3rd Qu.:11.10   3rd Qu.:6.000  
##  Max.   :14.90   Max.   :8.000
\end{verbatim}

\section{Univariate Plots Section}\label{univariate-plots-section}

\begin{quote}
\textbf{Tip}: In this section, you should perform some preliminary
exploration of your dataset. Run some summaries of the data and create
univariate plots to understand the structure of the individual variables
in your dataset. Don't forget to add a comment after each plot or
closely-related group of plots! There should be multiple code chunks and
text sections; the first one below is just to help you get started.
\end{quote}

\begin{quote}
\textbf{Tip}: Make sure that you leave a blank line between the start /
end of each code block and the end / start of your Markdown text so that
it is formatted nicely in the knitted text. Note as well that text on
consecutive lines is treated as a single space. Make sure you have a
blank line between your paragraphs so that they too are formatted for
easy readability.
\end{quote}

\section{Univariate Analysis}\label{univariate-analysis}

*\textbf{A} First, I made a plot to find out what it looks like as a
plot. I decided to pick quality for the univariate Analysis.

source:
\url{https://stackoverflow.com/questions/38788357/change-bar-plot-colour-in-geom-bar-with-ggplot2-in-r}

\paragraph{Plot a histogram}\label{plot-a-histogram}

\includegraphics{red-wine-data_files/figure-latex/unnamed-chunk-1-1.pdf}

Because my plot looks slightly skewed, I plan to transform it into a
normal distribution. I have two options: sqrt or log.

*source:
\url{https://stats.stackexchange.com/questions/74537/log-or-square-root-transformation-for-arima}

*\textbf{B)} This is my attempt of using sqrt to transform into a normal
distribution. It looks slightly normal.

\paragraph{Plot the histogram by adding
scale\_y\_sqrt()}\label{plot-the-histogram-by-adding-scale_y_sqrt}

\includegraphics{red-wine-data_files/figure-latex/Univariate-sqrt-normal-distribution-1.pdf}

*\textbf{C} This is my attempt of using log10 to transform into a normal
distribution. It looks like a perfect normal distribution.

\paragraph{Plot a histogram with
log10}\label{plot-a-histogram-with-log10}

\includegraphics{red-wine-data_files/figure-latex/Univariate-log10-normal-distribution-1.pdf}

\subsubsection{What is the structure of your
dataset?}\label{what-is-the-structure-of-your-dataset}

\begin{itemize}
\tightlist
\item
  Our dataset consists of 13 variables, with almost 1599 observations.
\end{itemize}

For more information, read {[}Cortez et al., 2009{]}.

Input variables (based on physicochemical tests): 1 - fixed acidity
(tartaric acid - g / dm\^{}3) 2 - volatile acidity (acetic acid - g /
dm\^{}3) 3 - citric acid (g / dm\^{}3) 4 - residual sugar (g / dm\^{}3)
5 - chlorides (sodium chloride - g / dm\^{}3 6 - free sulfur dioxide (mg
/ dm\^{}3) 7 - total sulfur dioxide (mg / dm\^{}3) 8 - density (g /
cm\^{}3) 9 - pH 10 - sulphates (potassium sulphate - g / dm3) 11 -
alcohol (\% by volume) Output variable (based on sensory data): 12 -
quality (score between 0 and 10)

\paragraph{Diaplay the dimmensions of an Object like in the
begining}\label{diaplay-the-dimmensions-of-an-object-like-in-the-begining}

\begin{verbatim}
## [1] 1599   13
\end{verbatim}

\subsubsection{What is/are the main feature(s) of interest in your
dataset?}\label{what-isare-the-main-features-of-interest-in-your-dataset}

\begin{itemize}
\tightlist
\item
  My main feature of interest is the quality of the wine.
\end{itemize}

\subsubsection{\texorpdfstring{What other features in the dataset do you
think will help support your\\
investigation into your feature(s) of
interest?}{What other features in the dataset do you think will help support your investigation into your feature(s) of interest?}}\label{what-other-features-in-the-dataset-do-you-think-will-help-support-your-investigation-into-your-features-of-interest}

\begin{itemize}
\tightlist
\item
  I believe that that I am interesting in all the variables such as
  volatile.acidity, residual.sugar, free.sulfur.dioxide, density,
  sulphates, fixed acidity, citric-acid, chlorides, pH, and alcohol.
\end{itemize}

\begin{verbatim}
## <ScaleContinuousPosition>
##  Range:  
##  Limits:    0 --    1
\end{verbatim}

\includegraphics{red-wine-data_files/figure-latex/unnamed-chunk-2-1.pdf}

\subparagraph{Here I have categorized each
plots}\label{here-i-have-categorized-each-plots}

\begin{itemize}
\item
  Right-Skewed: alcohol, citric acid, sulphates, Free sulfur dioxide,
  Fixed acidity, Total sulfur, chlorides
\item
  Symetric: density, PH, volatile.acidity, fixed acidity
\end{itemize}

\subsubsection{Did you create any new variables from existing variables
in the
dataset?}\label{did-you-create-any-new-variables-from-existing-variables-in-the-dataset}

Yes, I did create new variables to assembled all the plots in one box to
faciliate my observations.

\subsubsection{\texorpdfstring{Of the features you investigated, were
there any unusual distributions?\\
\#\#\#Did you perform any operations on the data to tidy, adjust, or
change the form\\
\#\#\#of the data? If so, why did you do
this?}{Of the features you investigated, were there any unusual distributions? \#\#\#Did you perform any operations on the data to tidy, adjust, or change the form \#\#\#of the data? If so, why did you do this?}}\label{of-the-features-you-investigated-were-there-any-unusual-distributions-did-you-perform-any-operations-on-the-data-to-tidy-adjust-or-change-the-form-of-the-data-if-so-why-did-you-do-this}

I used the log10 or squrt by transforming my plots into a normal
distribution. I am going to quote from r-statistics.com to explain why I
made them normal. ``The need for data transformation can depend on the
modeling method that you plan to use. For linear and logistic
regression, for example, you ideally want to make sure that the
relationship between input variables and output variables is
approximately linear, that the input variables are approximately normal
in distribution, and that the output variable is constant variance (that
is, the variance of the output variable is independent of the input
variables). You may need to transform some of your input variables to
better meet these assumptions.''

source:
\url{https://www.r-statistics.com/2013/05/log-transformations-for-skewed-and-wide-distributions-from-practical-data-science-with-r/}

\begin{center}\rule{0.5\linewidth}{\linethickness}\end{center}

\section{Bivariate Plots Section}\label{bivariate-plots-section}

I am going to use scatterplots to check the relationship between two
variables.

GGpairs can be useful for exploring the relationships between several
columns of data in a data frame

source:
\url{https://stackoverflow.com/questions/45044157/how-do-you-add-jitter-to-a-scatterplot-matrix-in-ggpairs}

\includegraphics{red-wine-data_files/figure-latex/Bivariate_Plots-1.pdf}
-Also, the citric acid do have connection to pH and density.

-The ones that are little bit closer to citric acid are volatile acidity
and fixed acidity. --Meaning, they are probably in relationship.

First, I need to double check by adding red line or linear regression to
see the connection connection between two supportive variables.

\paragraph{I am comparing pH and
density.}\label{i-am-comparing-ph-and-density.}

Density goes down while ph goes up\ldots{} I am surprized that they are
look very different to each other. I expected them to have a positive
correlation because they have normal plots. This plot goes to an
opposite dirrection.

\includegraphics{red-wine-data_files/figure-latex/unnamed-chunk-4-1.pdf}

Comparing ph and citric acid, it goes down.

\includegraphics{red-wine-data_files/figure-latex/unnamed-chunk-5-1.pdf}

Despite that density and Ph are both normal\ldots{}; it looks different
to citric acid ---comparing to pH and citric acid (check my previous
plot).

\includegraphics{red-wine-data_files/figure-latex/unnamed-chunk-6-1.pdf}

Here is the relationship between fixed acidity and ph. It is falling
down slightly.

\includegraphics{red-wine-data_files/figure-latex/unnamed-chunk-7-1.pdf}

Alright, I am going to compare fixed acidity and citric.acid. They are
SO connected to each other. Definately a positive one.

\includegraphics{red-wine-data_files/figure-latex/unnamed-chunk-8-1.pdf}

Next, I am going to plot volatile.acid and fixed.acidity. It definately
feels like it is going down.

\includegraphics{red-wine-data_files/figure-latex/unnamed-chunk-9-1.pdf}

Let's see how it looks like with pH and volatile.acidity. It definiately
looks positive because it is escalating.

\includegraphics{red-wine-data_files/figure-latex/unnamed-chunk-10-1.pdf}

Next, I am going to compare volatile.acidity and density\ldots{}It looks
like to me there is slightly increase from this plot.

\includegraphics{red-wine-data_files/figure-latex/unnamed-chunk-11-1.pdf}

This is how I categorized visually based on my previous plots.

\subparagraph{Positive correlation: (3
pos)}\label{positive-correlation-3-pos}

density \& citric. acidity, citric \& fixed acidity, volatile. acidity
\& pH, density \& volatile acidity.

\subparagraph{Negative correlation: (4
neg)}\label{negative-correlation-4-neg}

pH \& density, pH \& citric acidity, pH \& fixed acidity,
volatile.acidity \& fixed acidity.

\begin{center}\rule{0.5\linewidth}{\linethickness}\end{center}

Another way to verify if they are negative or positive correlation. I am
using a cor.test in programming.

\begin{verbatim}
## 
##  Pearson's product-moment correlation
## 
## data:  df$pH and df$density
## t = -14.53, df = 1597, p-value < 2.2e-16
## alternative hypothesis: true correlation is not equal to 0
## 95 percent confidence interval:
##  -0.3842835 -0.2976642
## sample estimates:
##        cor 
## -0.3416993
\end{verbatim}

\begin{verbatim}
## 
##  Pearson's product-moment correlation
## 
## data:  df$pH and df$citric.acid
## t = -25.767, df = 1597, p-value < 2.2e-16
## alternative hypothesis: true correlation is not equal to 0
## 95 percent confidence interval:
##  -0.5756337 -0.5063336
## sample estimates:
##        cor 
## -0.5419041
\end{verbatim}

\begin{verbatim}
## 
##  Pearson's product-moment correlation
## 
## data:  df$pH and df$fixed.acidity
## t = -37.366, df = 1597, p-value < 2.2e-16
## alternative hypothesis: true correlation is not equal to 0
## 95 percent confidence interval:
##  -0.7082857 -0.6559174
## sample estimates:
##        cor 
## -0.6829782
\end{verbatim}

\begin{verbatim}
## 
##  Pearson's product-moment correlation
## 
## data:  df$fixed.acidity and df$volatile.acidity
## t = -10.589, df = 1597, p-value < 2.2e-16
## alternative hypothesis: true correlation is not equal to 0
## 95 percent confidence interval:
##  -0.3013681 -0.2097433
## sample estimates:
##        cor 
## -0.2561309
\end{verbatim}

\begin{verbatim}
## 
##  Pearson's product-moment correlation
## 
## data:  df$pH and df$volatile.acidity
## t = 9.659, df = 1597, p-value < 2.2e-16
## alternative hypothesis: true correlation is not equal to 0
## 95 percent confidence interval:
##  0.1880823 0.2807254
## sample estimates:
##       cor 
## 0.2349373
\end{verbatim}

\begin{verbatim}
## 
##  Pearson's product-moment correlation
## 
## data:  df$density and df$citric.acid
## t = 15.665, df = 1597, p-value < 2.2e-16
## alternative hypothesis: true correlation is not equal to 0
## 95 percent confidence interval:
##  0.3216809 0.4066925
## sample estimates:
##       cor 
## 0.3649472
\end{verbatim}

\begin{verbatim}
## 
##  Pearson's product-moment correlation
## 
## data:  df$pH and df$volatile.acidity
## t = 9.659, df = 1597, p-value < 2.2e-16
## alternative hypothesis: true correlation is not equal to 0
## 95 percent confidence interval:
##  0.1880823 0.2807254
## sample estimates:
##       cor 
## 0.2349373
\end{verbatim}

Again the negative correlarion by using Pearson method in programming
are: - ph \& density - ph \& citrict acid - ph \& fixed acidity - fixed
acidity \& volatile-acidity

The positive correlation would be: -ph \& volatile -density \& fixed
acidity

\begin{longtable}[]{@{}l@{}}
\toprule
\begin{minipage}[t]{0.47\columnwidth}\raggedright\strut
-I am going to use ggcorr as an another method to check the
relationshipness between two variables.\strut
\end{minipage}\tabularnewline
\begin{minipage}[t]{0.47\columnwidth}\raggedright\strut
\includegraphics{red-wine-data_files/figure-latex/unnamed-chunk-13-1.pdf}\strut
\end{minipage}\tabularnewline
\begin{minipage}[t]{0.47\columnwidth}\raggedright\strut
I guess this method is easier because you can see that the bright red
color is indicating the strongest relationship between two
variables.These indicate the list of strongest relationship.\strut
\end{minipage}\tabularnewline
\begin{minipage}[t]{0.47\columnwidth}\raggedright\strut
1. Free sulfur dioxide \& total sulfur dioxide\strut
\end{minipage}\tabularnewline
\begin{minipage}[t]{0.47\columnwidth}\raggedright\strut
2. Fixed acidity \& Citric Adid\strut
\end{minipage}\tabularnewline
\begin{minipage}[t]{0.47\columnwidth}\raggedright\strut
3. density \& fixed acidity\strut
\end{minipage}\tabularnewline
\begin{minipage}[t]{0.47\columnwidth}\raggedright\strut
I always thought that ph and density would be in the list. Turns out
that their relationship is not the strongest. I need to keep my eyes on
the fixed acidity,\strut
\end{minipage}\tabularnewline
\begin{minipage}[t]{0.47\columnwidth}\raggedright\strut
\#\#\#\#\# Checking out the boxplots\strut
\end{minipage}\tabularnewline
\begin{minipage}[t]{0.47\columnwidth}\raggedright\strut
--\textgreater{} Quality of Wine and other variables in boxplots\strut
\end{minipage}\tabularnewline
\begin{minipage}[t]{0.47\columnwidth}\raggedright\strut
This time I am going to explore the data by displaying the scatterplots.
My main variable of my interest would be the quality of the wine. I am
going to observe the quality of the wine along with all other variables.
The question is what makes a red wine a good quality? or what define a
good quality of the red wine?\strut
\end{minipage}\tabularnewline
\begin{minipage}[t]{0.47\columnwidth}\raggedright\strut
\includegraphics{red-wine-data_files/figure-latex/unnamed-chunk-14-1.pdf}\strut
\end{minipage}\tabularnewline
\begin{minipage}[t]{0.47\columnwidth}\raggedright\strut
I am going to classify based on the previous observations.All these tiny
dots will represent the quality of the wine and remain the same.\strut
\end{minipage}\tabularnewline
\begin{minipage}[t]{0.47\columnwidth}\raggedright\strut
1. The quality of the wine gets better if we add more alcohol.\strut
\end{minipage}\tabularnewline
\begin{minipage}[t]{0.47\columnwidth}\raggedright\strut
2. I am suprized that we need to add more sulphates to improve the
quality of the wine.\strut
\end{minipage}\tabularnewline
\begin{minipage}[t]{0.47\columnwidth}\raggedright\strut
3. The more I add Citric acid, the better quality the wine will
be.{[}This is the one that I need to keep my eyes on this
variable.{]}\strut
\end{minipage}\tabularnewline
\begin{minipage}[t]{0.47\columnwidth}\raggedright\strut
4. The wine gets better with slight increase of the fixed acidity.\strut
\end{minipage}\tabularnewline
\begin{minipage}[t]{0.47\columnwidth}\raggedright\strut
5. The quality is falling down when the volatile acidity is
increased.\strut
\end{minipage}\tabularnewline
\begin{minipage}[t]{0.47\columnwidth}\raggedright\strut
6. The quality of the wine gets better when Ph goes up.\strut
\end{minipage}\tabularnewline
\begin{minipage}[t]{0.47\columnwidth}\raggedright\strut
7. The quality of the wine degrades when density goes up.\strut
\end{minipage}\tabularnewline
\begin{minipage}[t]{0.47\columnwidth}\raggedright\strut
8. Total Sulfur dioxids lowers the quality of the wine.\strut
\end{minipage}\tabularnewline
\begin{minipage}[t]{0.47\columnwidth}\raggedright\strut
9.Residual sugar does not change the quality of the wine --no matter how
much ammount you add sugar.\strut
\end{minipage}\tabularnewline
\begin{minipage}[t]{0.47\columnwidth}\raggedright\strut
\# Bivariate Analysis\strut
\end{minipage}\tabularnewline
\begin{minipage}[t]{0.47\columnwidth}\raggedright\strut
\#\#\# Talk about some of the relationships you observed in this part of
the\\
\#\#investigation. How did the feature(s) of interest vary with other
features in\\
\#\#the dataset?\strut
\end{minipage}\tabularnewline
\begin{minipage}[t]{0.47\columnwidth}\raggedright\strut
source:
\url{https://www.emathzone.com/tutorials/basic-statistics/positive-and-negative-correlation.html\#ixzz5OmFgYWlq}\strut
\end{minipage}\tabularnewline
\begin{minipage}[t]{0.47\columnwidth}\raggedright\strut
After observing the table, I see two similarity between density \&
citric acid and pH \& citrict acid. They both have normal looking
(symmetrical) plot. I have expected that these variables may have the
strongest relationship.\strut
\end{minipage}\tabularnewline
\begin{minipage}[t]{0.47\columnwidth}\raggedright\strut
After testing each different methods by plotting the variables with red
line, in Pearson method of finding correlation, and in spearman
correlation method --it turns out that\strut
\end{minipage}\tabularnewline
\begin{minipage}[t]{0.47\columnwidth}\raggedright\strut
-the fixed acidity \& density, -Free sulfur dioxide \& total sulfur
dioxide, -and fixed acidity \& citric acid\strut
\end{minipage}\tabularnewline
\begin{minipage}[t]{0.47\columnwidth}\raggedright\strut
have the strongest relationship.\strut
\end{minipage}\tabularnewline
\begin{minipage}[t]{0.47\columnwidth}\raggedright\strut
Therefore, my expectation is wrong.\strut
\end{minipage}\tabularnewline
\begin{minipage}[t]{0.47\columnwidth}\raggedright\strut
\#\#\# Did you observe any interesting relationships between the other
features\\
\#\#\#(not the main feature(s) of interest)?\strut
\end{minipage}\tabularnewline
\begin{minipage}[t]{0.47\columnwidth}\raggedright\strut
I have compared the ph, density, citric.accidity, and others variables
that might seem to be correlated to each other.\strut
\end{minipage}\tabularnewline
\begin{minipage}[t]{0.47\columnwidth}\raggedright\strut
\#\#\# What was the strongest relationship you found?\strut
\end{minipage}\tabularnewline
\begin{minipage}[t]{0.47\columnwidth}\raggedright\strut
The List of the strongest relationship:\strut
\end{minipage}\tabularnewline
\begin{minipage}[t]{0.47\columnwidth}\raggedright\strut
1. Free sulfur dioxide \& total sulfur dioxide\strut
\end{minipage}\tabularnewline
\begin{minipage}[t]{0.47\columnwidth}\raggedright\strut
2. Fixed acidity \& Citric Adid\strut
\end{minipage}\tabularnewline
\begin{minipage}[t]{0.47\columnwidth}\raggedright\strut
3. density \& fixed acidity\strut
\end{minipage}\tabularnewline
\bottomrule
\end{longtable}

\section{Multivariate Plots Section}\label{multivariate-plots-section}

\section{Multivariate Analysis}\label{multivariate-analysis}

\subsubsection{\texorpdfstring{Talk about some of the relationships you
observed in this part of the\\
\#\#\#investigation. Were there features that strengthened each other in
terms of\\
\#\#\#looking at your feature(s) of
interest?}{Talk about some of the relationships you observed in this part of the \#\#\#investigation. Were there features that strengthened each other in terms of \#\#\#looking at your feature(s) of interest?}}\label{talk-about-some-of-the-relationships-you-observed-in-this-part-of-the-investigation.-were-there-features-that-strengthened-each-other-in-terms-of-looking-at-your-features-of-interest}

During my investigation, the variables that i have been observing would
be density, alcohol, sulphates and quality of the wine.

According to my previous investigation, I am already aware that adding
more alcohol would improve the quality of the wine. I am going to
attempt to correlate with density (--a slight increase of the density
would makes the quality of the wine worse). Because I am curious. I want
to check the relationship between the density, alcohol, and quality.

We could conclude that adding less density, more alcohol could improve
the overall the quality of the wine. There is no contradiction from the
Bivariate examination.

\includegraphics{red-wine-data_files/figure-latex/unnamed-chunk-15-1.pdf}

Next one looks like a positive correlation\ldots{}It means that the
slight of the increase of the sulphites and alcohol could make better
quality. THerefore, there is no contradiction.

\includegraphics{red-wine-data_files/figure-latex/unnamed-chunk-16-1.pdf}

I am going to use another method and perform logistic method to check
the numbers. This time, I decided to seperate the quality by using
facet\_wrap because \ldots{}I am curious how it looks when the quality
are seperated.

Previously at the bivariate experimentation, the relationship between
the sulphates and quality are considered positive. However, adding
alcohol as a third variable change everythhing. Turns out that the
relationship between three variables like sulphates, alcohol, and
quality are tiny bit negative. Actually, that does not convince me that
the sulphates is making worse. I move on.

\includegraphics{red-wine-data_files/figure-latex/unnamed-chunk-17-1.pdf}

Previously from the bivariate experiementation with the quality and
citric acid. The quality from 3, 4, 6, and 8 would make a huge
contradiction. I guess I am going to use another method to re-examine
the relationship.

\includegraphics{red-wine-data_files/figure-latex/unnamed-chunk-18-1.pdf}

I have picked the variables the ones that had strong relationship from
the bivariate experimentation. I choose fixed acidity, citric acid, and
quality. These plots definately show a correlation.

\includegraphics{red-wine-data_files/figure-latex/unnamed-chunk-19-1.pdf}

I am going to attempt to check the linear models to make some
prediction.

source:
\url{https://stat.ethz.ch/R-manual/R-patched/library/base/html/numeric.html}

\begin{verbatim}
## 
## Calls:
## m1: lm(formula = as.numeric(quality) ~ alcohol, data = df)
## m2: lm(formula = as.numeric(quality) ~ alcohol + pH, data = df)
## m3: lm(formula = as.numeric(quality) ~ alcohol + pH + citric.acid, 
##     data = df)
## m4: lm(formula = as.numeric(quality) ~ alcohol + pH + citric.acid + 
##     volatile.acidity, data = df)
## m5: lm(formula = as.numeric(quality) ~ alcohol + pH + citric.acid + 
##     volatile.acidity + fixed.acidity, data = df)
## 
## ==========================================================================================
##                          m1            m2            m3            m4            m5       
## ------------------------------------------------------------------------------------------
##   (Intercept)          -0.125         2.426***      1.232**       2.672***      1.751**   
##                        (0.175)       (0.387)       (0.460)       (0.457)       (0.574)    
##   alcohol               0.361***      0.386***      0.364***      0.334***      0.334***  
##                        (0.017)       (0.017)       (0.017)       (0.017)       (0.017)    
##   pH                                 -0.850***     -0.463**      -0.529***     -0.329*    
##                                      (0.116)       (0.141)       (0.135)       (0.155)    
##   citric.acid                                       0.521***     -0.180        -0.361**   
##                                                    (0.110)       (0.121)       (0.138)    
##   volatile.acidity                                               -1.361***     -1.409***  
##                                                                  (0.113)       (0.114)    
##   fixed.acidity                                                                 0.040**   
##                                                                                (0.015)    
## ------------------------------------------------------------------------------------------
##   R-squared             0.227         0.252         0.262         0.324         0.327     
##   adj. R-squared        0.226         0.251         0.261         0.322         0.325     
##   sigma                 0.710         0.699         0.694         0.665         0.664     
##   F                   468.267       268.888       189.108       190.704       154.539     
##   p                     0.000         0.000         0.000         0.000         0.000     
##   Log-likelihood    -1721.057     -1694.466     -1683.339     -1613.978     -1610.469     
##   Deviance            805.870       779.508       768.735       704.854       701.767     
##   AIC                3448.114      3396.931      3376.678      3239.957      3234.938     
##   BIC                3464.245      3418.440      3403.564      3272.220      3272.578     
##   N                  1599          1599          1599          1599          1599         
## ==========================================================================================
\end{verbatim}

According to my observation, if I check the result of the r-square and
intercept. I find out that alcohol + pH + citric.acid + volatile.acidity
make a great wine quality. However, alcohol + pH + citric.acid is
degrading. I believe that adding the citric acid is the cause of
dimininshing the quality of the wine.

\subsubsection{Were there any interesting or surprising interactions
between
features?}\label{were-there-any-interesting-or-surprising-interactions-between-features}

Just like I was expected that the multivariate would have different
results from univariate or bivariate examination. I am suprised that I
find out something that contradicts from my bivariate investigation.
Sometimes, you might remind yourself that Simpson paradox is everywhere.
I learned my lesson that I need to further the experimentation to verify
again the relationship in the multivariate experiment.

\subsubsection{\texorpdfstring{OPTIONAL: Did you create any models with
your dataset? Discuss the\\
\#\#strengths and limitations of your
model.}{OPTIONAL: Did you create any models with your dataset? Discuss the \#\#strengths and limitations of your model.}}\label{optional-did-you-create-any-models-with-your-dataset-discuss-the-strengths-and-limitations-of-your-model.}

\subsection{I made a linear models to check the result in numbers
instead of the plots. The strenght is that I find it easier to read the
numbers on the table than plots. We know what causes the increase and
decrease of the wine quality. The limitation of my model is that is hard
to predict what makes bad and good quality. Also, all my p-values are
all 0\ldots{} meaning it is harder to interpret the confident interval
in this experimentation.(I am 0\%
confident\ldots{})}\label{i-made-a-linear-models-to-check-the-result-in-numbers-instead-of-the-plots.-the-strenght-is-that-i-find-it-easier-to-read-the-numbers-on-the-table-than-plots.-we-know-what-causes-the-increase-and-decrease-of-the-wine-quality.-the-limitation-of-my-model-is-that-is-hard-to-predict-what-makes-bad-and-good-quality.-also-all-my-p-values-are-all-0-meaning-it-is-harder-to-interpret-the-confident-interval-in-this-experimentation.i-am-0-confident}

\section{Final Plots and Summary}\label{final-plots-and-summary}

Here are my three I like and find helpful to understand what makes a
good and bad quality of the wine.

\subsubsection{Plot One}\label{plot-one}

\begin{verbatim}
## `stat_bin()` using `bins = 30`. Pick better value with `binwidth`.
## `stat_bin()` using `bins = 30`. Pick better value with `binwidth`.
## `stat_bin()` using `bins = 30`. Pick better value with `binwidth`.
## `stat_bin()` using `bins = 30`. Pick better value with `binwidth`.
## `stat_bin()` using `bins = 30`. Pick better value with `binwidth`.
## `stat_bin()` using `bins = 30`. Pick better value with `binwidth`.
## `stat_bin()` using `bins = 30`. Pick better value with `binwidth`.
## `stat_bin()` using `bins = 30`. Pick better value with `binwidth`.
## `stat_bin()` using `bins = 30`. Pick better value with `binwidth`.
## `stat_bin()` using `bins = 30`. Pick better value with `binwidth`.
## `stat_bin()` using `bins = 30`. Pick better value with `binwidth`.
\end{verbatim}

\includegraphics{red-wine-data_files/figure-latex/Plot_One-1.pdf}

\subsubsection{Description One}\label{description-one}

The main reason why I pick this plot because it helps me to observe the
single variable histograms very quickly. I could easily categorize which
one is a normal distribution.

Again , I concluded that combining with citric acid dimish the quality
of the wine with alcohol.

There are two things we might need to re-observe is the citric acid and
alcohol. Both have long looking negative skewed plots. I was wrong to
pressume that the alcohol and citric acid could make the wine better
because of their similariy.

\subsubsection{Plot Two}\label{plot-two}

\includegraphics{red-wine-data_files/figure-latex/unnamed-chunk-21-1.pdf}

\subsubsection{Description Two}\label{description-two}

The target is to check the correlation between two variables. Labeling
with colors and numbers are really quick way to check what the
relationships. For example, the red color indicated that the two
variables have strong relationship.

Let's rexamine the citric acid and alcohol. We could get an idea that it
has a slight correlation of 0.1. (not too strong or low)

\subsubsection{Plot Three}\label{plot-three}

\begin{verbatim}
## Warning: Transformation introduced infinite values in continuous y-axis

## Warning: Transformation introduced infinite values in continuous y-axis
\end{verbatim}

\begin{verbatim}
## Warning: Removed 132 rows containing non-finite values (stat_smooth).
\end{verbatim}

\includegraphics{red-wine-data_files/figure-latex/Plot_Three-1.pdf}

\subsubsection{Description Three}\label{description-three}

I choose this one because I want to point out how citric acid , alcohol,
and quality are very different when we examine in multivariate
experimentation. I clearly see that the quality 6 and 7 are in slightly
better. Overall, that does not convince me that the citric acid is
making a better quality because most of the report from quality 3,4,5,
and 8 indicate that the wine got worse. \# Reflection

At first, I have concluded that adding more citric acid in wine would
improve the quality of the wine. I was not sure if I should conclude
that citric acid is one that is decreasing of the quality because
majority of the plots with citric acid skewed positively.

As I continue the research and use the linear model, my view about the
citric acid, alcohol and quality has changed. Citric acid is definately
diminishing the quality of the wine with alcohol.

I have suspected since in the begining that there is some Simpson
paradox moment going on during the examination. As I progress, I kept
being skeptical about what I saw. I started to understand that I have to
further the research from univariate, bivariate, and multivariate to
examine the differences of the variables relationship. It is almost like
having one child in the family can affect two parents's life. I find it
really facinating.I would suggest to re-examine again other variables
beside alcohol and quality in multivariate examination.

To improve the research, I would love to interpret the confidence
interval and have the numbers in the p-values. I think the research
would be strong and convincinble if I include ``I am 95\%
confident\ldots{}''


\end{document}
